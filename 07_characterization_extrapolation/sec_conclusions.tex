In this chapter, we presented the results of numerical simulations conducted in order to characterize and compare the performance of the plane-wave and ambisonics translation methods.
Following the simulation framework laid out in \chapref{chap:06_Simulation_Framework}, we simulated simple incident sound fields consisting of a single microphone and a single point-source, varying source distance and azimuth, as well as microphone distance and listener position.
First, in \secref{sec:07_Characterization_Extrapolation:Plane-wave_Dependence}, we determined suitable parameters for the plane-wave translation method, comparing the beamforming and pseudoinversion methods for computing the plane-wave decomposition and varying both the ambisonics input order and the number of plane-wave terms.
We then explored, in \secref{sec:07_Characterization_Extrapolation:Azimuth_Dependence}, basic properties of each method by computing the effective frequency responses induced by the plane-wave and ambisonics translation methods across source azimuths.
Finally, in \secref{sec:07_Characterization_Extrapolation:Results}, we conducted a more comprehensive analysis of both methods in terms of the metrics enumerated in \secref{sec:06_Simulation_Framework:Metrics} for sound level, spectral coloration, source localization, and diffuseness.

The analyses presented in this chapter yielded the following major findings:
\begin{itemize}
\item for the plane-wave translation method, a clear advantage exists to using the beamforming plane-wave decomposition method (see \eqnref{eq:02_Acoustical_Theory:A2mu}) and matching the number of ambisonics signals, $N$, to the number of plane-wave terms, $Q$;%
\footnote{All subsequent findings relate to this implementation of the plane-wave translation method: beamforming with matched $Q = N$.}
%\item the pseudoinversion plane-wave decomposition method (see \eqnref{eq:02_Acoustical_Theory:A2mu_Pinv}) exhibited a similar advantage, as well as an advantage to ``oversampling'' directions by taking $Q > N$, although this method also appeared more sensitive to mismatches than the beamforming method;
\item the frequency responses induced by the plane-wave translation method are largely flat but with sporadic notches while those induced by the ambisonics translation method exhibit a consistent low-pass-like roll-off of high-frequency energy, but both methods appear largely insensitive to source azimuth;
\item the ambisonics translation method incurs significant errors in both level and coloration at all source distances which, overall, increase steadily with microphone distance;
\item for exterior sources, the plane-wave translation method achieves a high degree of accuracy in both level and localization;
\item for interior sources, where the region of validity restriction is violated, both methods incur significant errors in both level and localization; and
\item more generally, both methods tend to exhibit two distinct regimes of behavior for exterior and interior sources, with a transition region between the two, and the performance for interior sources is often degraded compared to that for exterior sources.
\end{itemize}
Additionally, results showed that increasing the ambisonics order tends to uniformly improve the performance of both methods, with the exception of the spectral errors incurred by the plane-wave translation method.

Due to the extremely large level and coloration errors incurred by the ambisonics translation method, the plane-wave translation method is likely the only viable method for most applications with practical translation distances (e.g., $u > 0.5$~m).
This method is particularly well-suited for exterior sources, but its performance degrades significantly, in terms of localization in particular, once the region of validity restriction is violated.
The remaining challenge for this method is the introduction of significant spectral errors ($\rho_\eta > 10$~dB) over all microphone and source distances.
Consequently, future improvements to this method should attempt to correct this coloration, perhaps parametrically based on source direction.

As demonstrated in this chapter, the performance of linear extrapolation-based navigational methods is significantly impaired by the presence of near-field sources.
Consequently, parametric and interpolation-based approaches have been developed that aim to extend navigation beyond such near-field sources while maintaining acceptable performance.
In the following chapters, we characterize and compare the performance of several such methods:
in \chapref{chap:08_Proposed_Method}, we propose a parametric interpolation method and demonstrate its improvement over a benchmark linear interpolation method and,
in \chapref{chap:09_Thiergart_Comparison}, we perform a similar analysis of an existing parametric interpolation method.