Recently, \citet{Stitt2016} proposed an extension to incorporate the precedence effect into the energy vector of \citet{Gerzon1992}.
In their paper, \citeauthor{Stitt2016} also showed that their proposed model achieves improved localization prediction accuracy compared to the binaural models of \citet{Dietz2011} and \citet{Lindemann1986a}.
Motivated by these results, we proposed in a recent paper \citep{TylkaChoueiri2016} an extension to this localization model; here, we describe and revise our proposed extension.

We begin by converting the ambisonics impulse response into a set of $Q$ impulse responses for a specified grid of plane-wave directions, $\hat{v}_q$, with $q \in [1,Q]$.
For ambisonics order $L$, there are $N = (L + 1)^2$ ambisonics signals, $a_n(t)$, with $n \in [0, N-1]$, and the corresponding plane-wave signals, $\mu$, are given by \eqnref{eq:02_Acoustical_Theory:A2mu}.%
\footnote{Alternatively, this plane-wave decomposition could be computed using a pseudoinversion approach, as described in \eqnref{eq:02_Acoustical_Theory:A2mu_Pinv}, or a compressed sensing approach, as described by \citet[Eq.~(10)]{Epain2009}.}
The grids of directions (also called ``nodes'') used here are given by \citet{FliegeMaier1999}; these grids and their corresponding quadrature weights are freely available online.\citefooturl{FliegeNodesURL}
As this discrete plane-wave sound field would generally be rendered via quadrature integration (see \eqnref{eq:02_Acoustical_Theory:PW_Quadrature_Rendered_Field}), the impulse response for each plane-wave term is given by $g_q(t) = w_q \mu(t,\hat{v}_q)$, where $w_q$ is the quadrature weight for the direction $\hat{v}_q$.

Next, we identify and isolate temporally-distinct impulse response ``wavelets.''
To do this, we apply a $4^\textrm{th}$-order Butterworth high-pass filter with a cut-off frequency of 500~Hz to all impulse responses in the set and compute the global maximum (i.e., the largest absolute value over all impulse responses in the set).
For each impulse response, we take the absolute value and identify any peaks (i.e., local maxima) whose amplitudes are at least $G_\text{min}$~dB relative to the global maximum ($G_\text{min} \leq 0$~dB).
If no such peaks exist in a given impulse response, then that response in its entirety is treated as a wavelet.
If at least one such peak exists, then, around each peak, we apply a Tukey window beginning $\tau_\text{w}$~ms before the peak and ending either $\tau_\text{w}$~ms after the peak, or at the position of the following peak, whichever yields a larger window length.
Both the cosine fade-in and fade-out of the Tukey window are $\tau_\text{w}$~ms in duration.
In this way, a single impulse response may be split into several wavelets.%
\footnote{A simpler implementation of this stage of the model might follow a more traditional, uniformly-partitioned STFT approach, but we do not explore this idea here.}
For each wavelet, we apply a 10\% ($-20$~dB, now relative to the peak of that particular wavelet) threshold to determine the time-delay of the onset.

For the purposes of this model, we consider each wavelet to be a distinct sound source, such that wavelets extracted from the same impulse response originate from the same direction, but arrive at different times, given by their onset times.
Taking the Fourier transform of each wavelet yields complex-valued, frequency-dependent gains, $G_{q'}$, of the $q'{}^\text{th}$ wavelet, where $q' \in [1,Q']$ and $Q' \geq Q$ is the total number of wavelets.
We then average these gains in critical bands using a gammatone filter bank (see \figref{fig:04_Auditory_Models:Gammatone_Filters}),
such that the frequency-averaged gains are given by
\begin{equation}\label{eq:05_Proposed_Models:Localization_Model:FreqAvg_Gain}
\overline{G}_{q'}(f_c) = \frac{\displaystyle \int_{-\infty}^\infty |H_\Gamma(f;f_c)| |G_{q'}(f)| df}{\displaystyle \int_{-\infty}^\infty |H_\Gamma(f;f_c)| df},
\end{equation}
where the gammatone filters are taken for a set of ERB-spaced center frequencies spanning the range $f \in [20~\text{Hz}, 20~\text{kHz}]$.

For each frequency band, we feed these gains into the model, yielding a frequency-dependent predicted localization vector $\vec{\nu}_\textrm{PE}(f_c)$, given by
\begin{equation}\label{eq:05_Proposed_Models:PE_Energy_Vector}
\vec{\nu}_{\textrm{PE}}(f_c) = \frac{ \sum_{q'} |w_{q'}(\alpha) \overline{G}_{q'}(f_c)|^2 \, \hat{v}_{q'}}{ \sum_{q'} |w_{q'}(\alpha) \overline{G}_{q'}(f_c)|^2},
\end{equation}
where $\alpha$ is a parameter of the original model of \citet{Stitt2016} (discussed in \secref{sec:04_Auditory_Models:PE_Energy_Vector}) and $w_{q'}(\alpha)$ is a perceptual weight (based on the precedence-effect) for the $q'{}^\text{th}$ wavelet.
Similarly, we define a perceptually-weighted velocity vector, given by
\begin{equation}\label{eq:PE_Velocity_Vector}
\vec{\nu}_{\textrm{PV}}(f_c) = \text{Re} \left[ \frac{ \sum_{q'} w_{q'}(\alpha) \overline{G}_{q'}(f_c) \, \hat{v}_{q'}}{ \sum_{q'} w_{q'}(\alpha) \overline{G}_{q'}(f_c)} \right],
\end{equation}
where $\text{Re}(\cdot)$ denotes taking the real part of the argument.
Note that we include this step of ``taking the real part'' in our definition only to maintain consistency with the original definition of the velocity vector, given by \citet[section 4]{Gerzon1992}, in which complex-valued gains are used (see \eqnref{eq:04_Auditory_Models:Velocity_Vector}).
However, in our case, all $\overline{G}_{q'}$ are already real-valued (see \eqnref{eq:05_Proposed_Models:Localization_Model:FreqAvg_Gain}), so taking the real part has no effect on the result.
This deviation from the original definition may be justified since the time-dependence of the sources is now captured in the precedence-effect-based weights, so each source's phase is no longer needed.

We then combine the velocity vector below 700~Hz and the energy vector above into a single, frequency-dependent vector, given by
\begin{equation}
\vec{\nu}_{\textrm{PC}}(f_c) =
\begin{cases}
\displaystyle \frac{\left\| \vec{\nu}_{\textrm{PE}}(f_\text{XO}) \right\|}{\left\| \vec{\nu}_{\textrm{PV}}(f_\text{XO}) \right\|} \cdot \vec{\nu}_{\textrm{PV}}(f_c), & \text{for}~f_c \leq f_\text{XO}, \\
\vec{\nu}_{\textrm{PE}}(f_c), & \text{for}~f_c > f_\text{XO},
\end{cases}
\end{equation}
where $\|\cdot\|$ denotes the $\ell^2$ norm (Euclidean distance) of a vector, $f_\text{XO}$ is the ``crossover'' frequency, equal to the center frequency nearest to 700~Hz, and we have introduced a normalization factor to match low- and high-frequency vector magnitudes.
Finally, we compute a weighted-average vector, which depends on the stimulus signal and is given by
\begin{equation}
\vec{\nu}_{\textrm{P}} = \frac{\sum_c \overline{X}(f_c)\, \vec{\nu}_{\textrm{PC}}(f_c)}{\sum_c \overline{X}(f_c)},
\end{equation}
where the weights $\overline{X}(f_c)$ are the stimulus signal's energy in each critical band, given by
\begin{equation}
\overline{X}(f_c) = \frac{\displaystyle \int_{-\infty}^\infty |H_\Gamma(f;f_c)| |X(f)|^2 df}{\displaystyle \int_{-\infty}^\infty |H_\Gamma(f;f_c)| df},
\end{equation}
and $X(f)$ is the Fourier transform of the stimulus signal.

In addition to the three free model parameters ($Q$, $\tau_\text{w}$, and $G_\text{min}$) defined above, the original model of \citet{Stitt2016} retains one free parameter, $\alpha \in [0,1]$, which specifies the relative importance of stationary (i.e., time-averaged) to transient information in the stimulus signal (see \secref{sec:04_Auditory_Models:PE_Energy_Vector}).
As will be described below in \secref{sec:05_Proposed_Models:Localization_Results}, we determined optimal values for these four parameters based on a best fit of the model's predictions to the data from the listening experiment.