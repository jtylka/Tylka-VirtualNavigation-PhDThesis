In this work, we followed the approach of \citet{Wittek2007} and developed linear regression models to predict subjective ratings of coloration given some combination of the coloration metrics reviewed in \secref{sec:04_Auditory_Models:Coloration_Metrics}.
As discussed in \chapref{chap:04_Auditory_Models}, we define each of these metrics \textit{relative} to some reference signal.
Consequently, each metric is computed using both a \textit{test sample} (i.e., the ambisonics impulse response for the listening position after processing through some navigational method) and a \textit{reference sample} (i.e., the ambisonics impulse response captured directly at the listening position).

In this work, we used the combinations of coloration metrics listed in \tabref{tab:Coloration_Models} to create multiple linear regression models, each of which predicts subjective ratings of coloration.
Regression coefficients were found by fitting the predictions of each model to the data, as will be discussed below in \secref{sec:05_Proposed_Models:Coloration_Results}.

% Table of coloration models and corresponding metrics
\begin{table}[t]
\centering
 \begin{tabular}{|c|c|} \hline
 \textbf{Model Name} & \textbf{Metrics Used} \\ \hline
 Proposed & $\rho_\eta, \sigma_\eta, E_\text{pk}, E_\text{n}$ \\
 \citet{Kates1984} & $\rho_{e_\text{CS}}, \sigma_{e_\text{CS}}$ \\
 \citet{Pulkki1999} & $\rho_{e_\text{CLL}}, \sigma_{e_\text{CLL}}$ \\
 \citet{Wittek2007} & $\rho_{e_\text{IS}}, \sigma_{e_\text{IS}}$ \\ \hline
 \end{tabular}
 \caption{Metrics used for each coloration model.}
 \label{tab:Coloration_Models}
\end{table}