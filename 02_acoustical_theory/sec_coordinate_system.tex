As is common in ambisonics, we adopt Cartesian and spherical coordinate systems in which, for a listener positioned at the origin, the $+x$-axis points forward, the $+y$-axis points to the left, and the $+z$-axis points upward.
The point $(x,y,z)$ in Cartesian coordinates is given in spherical coordinates by
\begin{equation}\label{eq:02_Acoustical_Theory:Spherical_Coordinates}
\begin{aligned}
    r &= \sqrt{x^2 + y^2 + z^2},\\
    \theta &= \arcsin \frac{z}{r},\\ % = \arcsin \mu
    \phi &= \arctan \frac{y}{x}.
\end{aligned}
\end{equation}
From this definition we see that $r$ is the (nonnegative) radial distance from the origin, $\theta \in [-\pi/2, \pi/2]$ is the elevation angle above the horizontal ($x$-$y$) plane, and $\phi \in [0, 2\pi)$ is the azimuthal angle around the vertical ($z$) axis, with $(\theta,\phi) = (0,0)$ corresponding to the $+x$ direction and $(0,\pi/2)$ to the $+y$ direction.
These variables are defined graphically in~\figref{fig:02_Acoustical_Theory:Coordinate_System}.
Conversion back to Cartesian coordinates is given by
\begin{equation}\label{eq:02_Acoustical_Theory:Cartesian_Coordinates}
\begin{aligned}
    x &= r \cos \theta \cos \phi,\\
    y &= r \cos \theta \sin \phi,\\
    z &= r \sin \theta.
\end{aligned}
\end{equation}
For a position vector $\vec{r} = (x,y,z)$, we denote the corresponding unit vector by $\hat{r} \equiv \vec{r}/r$.

\tdplotsetmaincoords{60}{120}

\pgfmathsetmacro{\rvec}{.8}
\pgfmathsetmacro{\thetavec}{30}
\pgfmathsetmacro{\phivec}{60}

\begin{figure}[t]
\centering
\begin{tikzpicture}[scale=5,tdplot_main_coords]

    \coordinate (O) at (0,0,0);
    \draw[thick,->] (0,0,0) -- (1,0,0) node[anchor=north east]{$x$};
    \draw[thick,->] (0,0,0) -- (0,1,0) node[anchor=north west]{$y$};
    \draw[thick,->] (0,0,0) -- (0,0,1) node[anchor=south]{$z$};
    \tdplotsetcoord{P}{\rvec}{\thetavec}{\phivec}
    \draw[thick,->] (O) -- (P) node[above right] {$\vec{r}$};
    \draw[dashed] (O) -- (Pxy);
    \draw[dashed] (P) -- (Pxy);
    \tdplotdrawarc{(O)}{0.2}{0}{\phivec}{anchor=north}{$\phi$}
    \tdplotsetthetaplanecoords{\phivec}
    \tdplotdrawarc[tdplot_rotated_coords]{(0,0,0)}{0.2}{90}{\thetavec}{anchor=south west}{$\theta$}

\end{tikzpicture}
\caption{Coordinate system.}\label{fig:02_Acoustical_Theory:Coordinate_System}
\end{figure}

%% Short version for papers
%As is common in ambisonics, we adopt Cartesian and spherical coordinate systems in which, for a listener positioned at the origin, the $+x$-axis points forward, the $+y$-axis points to the left, and the $+z$-axis points upward.
%Correspondingly, $r$ is the (nonnegative) radial distance from the origin, $\theta \in [-\pi/2, \pi/2]$ is the elevation angle above the horizontal ($x$-$y$) plane, and $\phi \in [0, 2\pi)$ is the azimuthal angle around the vertical ($z$) axis, with $(\theta,\phi) = (0,0)$ corresponding to the $+x$ direction and $(0,\pi/2)$ to the $+y$ direction.
%For a position vector $\vec{r} = (x,y,z)$, we denote the corresponding unit vector by $\hat{r} \equiv \vec{r}/r$.