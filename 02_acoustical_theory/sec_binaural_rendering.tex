Generally, binaural decoding of ambisonics aims to generate the appropriate binaural signals for a listener at a point (with any orientation) in a recorded (or synthesized) sound field,
such that the resulting perception of the sound field is identical to that which would have occurred in the real sound field.
At the origin, this task may be accomplished by:
\begin{enumerate}
\item decoding the ambisonics signals to a fixed set of virtual loudspeakers and filtering each loudspeaker's signal by the appropriate HRTF \citep{McKeagMcGrath1996,Noisternig2003a},
\item transforming the ambisonics signals into a plane-wave expansion of the sound field and filtering each plane-wave term by the appropriate HRTF \citep{Duraiswami2005a,MenziesAlAkaidi2007b}, 
\item using a spherical-harmonic decomposition of the listener's HRTFs to filter each ambisonics signal directly \citep{RafaelyAvni2010,Bernschutz2014}, or
\item decomposing the (typically first-order only) ambisonics signals into directional (and diffuse) components using non-linear, parametric techniques (e.g., HARPEX, DirAC) and filtering each component by the appropriate HRTF \citep{Laitinen2008,BergeBarrett2010b}.
\end{enumerate}
In this section, we review the first three approaches listed above.
A corresponding MATLAB toolkit developed by the present author which implements these first three approaches is presented in \apxref{chap:A2_SABRE_Toolkit}.

\subsection{Virtual ambisonics}\label{sec:02_Acoustical_Theory:VA_Binaural}
Generally, the ambisonics decoding matrix, $\mathbf{D}$, which is typically time- and frequency-independent,%
\footnote{Although two- or multi-band decoders are commonly used, the decoder matrix for each sub-band is typically time-independent \citep{Heller2012}.}
determines the loudspeaker signals, $g_q$, given by
\begin{equation}\label{eq:02_Acoustical_Theory:Ambisonics_Decoding}
\mathbf{g}(t) = 
\begin{bmatrix}
g_{1}(t) \\ g_{2}(t) \\ \vdots \\ g_{Q}(t)
\end{bmatrix}
 = \mathbf{D} \cdot 
\begin{bmatrix}
a_{0}(t) \\ a_{1}(t) \\ \vdots \\ a_{N-1}(t)
\end{bmatrix}
 = \mathbf{D} \cdot \mathbf{a}(t),
\end{equation}
where $Q$ is the number of loudspeakers and $a_n$ is the $n^\text{th}$ ambisonics signal.
Methods of calculating the decoding matrix have been extensively researched and it is outside of the scope of this thesis to discuss them in detail.
Briefly, the decoding matrix attempts to use all available loudspeakers to create a perceptually-accurate reproduction the recorded sound field \citep{Heller2012}.

Following traditional ambisonic theory,
we consider the ambisonics signals produced at the center of the loudspeaker array
in response to the loudspeaker signals, given by
\begin{equation}
a_n(t) = \sum_{q=1}^{Q} Y_{n}(\hat{v}_q) g_q(t).
\end{equation}
(Note that this effectively treats the loudspeakers as plane-wave sources via \eqnref{eq:02_Acoustical_Theory:PlaneSource_An}.)
Equivalently, in matrix form, we have $\mathbf{a}(t) = \mathbf{Y} \cdot \mathbf{g}(t)$, where
\begin{equation}\label{eq:YMatrix}
\mathbf{Y} = 
\begin{bmatrix}
Y_{0}(\hat{v}_1) & Y_{0}(\hat{v}_2) & \cdots & Y_{0}(\hat{v}_Q) \\
Y_{1}(\hat{v}_1) & Y_{1}(\hat{v}_2) & \cdots & Y_{1}(\hat{v}_Q) \\
\vdots & \vdots & \ddots & \vdots \\
Y_{N-1}(\hat{v}_1) & Y_{N-1}(\hat{v}_2) & \cdots & Y_{N-1}(\hat{v}_Q)
\end{bmatrix}.
\end{equation}
From this formulation, we obtain the basic (pseudoinverse) decoder, given by \citep[Appendix~A.1]{Heller2008}
\begin{equation}\label{eq:02_Acoustical_Theory:PinvDecoder}
\mathbf{D} = \mathbf{Y}^{+},
\end{equation}
where $(\cdot)^{+}$ denotes pseudoinversion.
For a thorough review of ambisonic decoding theory and practice,
the reader is referred to the works of \citeauthor{Heller2008}~\citep{Heller2008,Heller2012}.

Assuming a free field and ideal loudspeakers that are equidistant from the listener,
the resulting binaural pressure signals are given by
\begin{equation}\label{eq:02_Acoustical_Theory:VA_Binaural}
p^{\text{L,R}}(t) = \left( \mathbf{h}^{\text{L,R}} \ast \mathbf{g} \right)(t)
 = \sum_{q=1}^Q \left( h_{q}^{\text{L,R}} \ast g_{q} \right) (t),
\end{equation}
where `$\ast$' denotes convolution (which, for matrices and vectors operates like matrix multiplication except each pair of elements is convolved rather than multiplied) and
\begin{equation}
\mathbf{h}^{\text{L,R}}(t) = 
\begin{bmatrix}
h_{1}^{\text{L,R}}(t) & h_{2}^{\text{L,R}}(t) & \cdots & h_{Q}^{\text{L,R}}(t)
\end{bmatrix}
\end{equation}
is a vector containing the head-related impulse responses (HRIRs) for a given listener
and for the directions of each loudspeaker.
The superscript ``$\text{L,R}$'' denotes the response at the left or right ear, respectively.
Combining \eqnreftwo{eq:02_Acoustical_Theory:Ambisonics_Decoding}{eq:02_Acoustical_Theory:VA_Binaural},
we obtain a simple matrix expression for rendering ambisonics to binaural, given by
\begin{equation}\label{eq:02_Acoustical_Theory:VA_Binaural_Matrix}
p^{\text{L,R}}(t) = \left( \mathbf{h}^{\text{L,R}} \ast \left( \mathbf{D} \cdot \mathbf{a} \right) \right)(t).
\end{equation}

\subsection{Plane-wave rendering}\label{sec:02_Acoustical_Theory:PW_Quadrature_Binaural}
For this rendering approach, we first compute the signature function, $\mu$, of the sound field,
as given by \eqnref{eq:02_Acoustical_Theory:A2mu}.
Equivalently, written as a matrix equation, we have
\begin{equation}\label{eq:02_Acoustical_Theory:A2mu_Matrix}
\mathbf{m}(t) = 
\begin{bmatrix}
\mu(t,\hat{v}_1) \\ \mu(t,\hat{v}_2) \\ \vdots \\ \mu(t,\hat{v}_Q)
\end{bmatrix} 
= \mathbf{Y}^{\textrm{T}} \cdot \mathbf{F}^{-1} \cdot \mathbf{a}(t),
\end{equation}
where $Q$ is now the number of plane-wave terms and $\mathbf{F}$ is a diagonal matrix given by
\begin{equation}
\mathbf{F} = \text{diag} \left\{ \begin{bmatrix} \left\|Y_0\right\|^{2} & \left\|Y_1\right\|^{2} & \cdots & \left\|Y_{N-1}\right\|^{2} \end{bmatrix} \right\}.
\end{equation}

Given the signature function, the binaural pressure signals can be approximately reconstructed using a finite number of plane-waves, given by \citep{Duraiswami2005a}
\begin{equation}\label{eq:02_Acoustical_Theory:PW_Quadrature_Binaural}
p^{\text{L,R}}(t) \approx \sum_{q=1}^Q \left( h_{q}^{\text{L,R}} \ast \left(w_q \mu(\hat{v}_q) \right) \right) (t)
 = \left(\mathbf{h}^{\text{L,R}} \ast \left( \mathbf{W} \cdot \mathbf{m} \right) \right)(t),
\end{equation}
where $w_q$ is the quadrature weight of the $q^\textrm{th}$ plane-wave term and we have let
\begin{equation}\label{eq:02_Acoustical_Theory:PW_Quadrature_Weights}
\mathbf{W} = \text{diag} \left\{ \begin{bmatrix} w_1 & w_2 & \cdots & w_Q \end{bmatrix} \right\}.
\end{equation}
Combining \eqnreftwo{eq:02_Acoustical_Theory:A2mu_Matrix}{eq:02_Acoustical_Theory:PW_Quadrature_Binaural}, we obtain
\begin{equation}\label{eq:02_Acoustical_Theory:PW_Quadrature_Binaural_Matrix}
p^{\text{L,R}}(t) = \left( \mathbf{h}^{\text{L,R}} \ast \left( \mathbf{W} \cdot \mathbf{Y}^{\textrm{T}} \cdot \mathbf{F}^{-1} \cdot \mathbf{a} \right) \right) (t).
\end{equation}

It is worth noting that, instead of using \eqnref{eq:02_Acoustical_Theory:A2mu_Matrix}, we can compute $\mu$ following the pseudoinverse approach of \eqnref{eq:02_Acoustical_Theory:PinvDecoder}, such that
\begin{equation}\label{eq:02_Acoustical_Theory:A2mu_Pinv}
\mathbf{m}(t) = \mathbf{W}^{-1} \cdot \mathbf{Y}^{+} \cdot \mathbf{a}(t).
\end{equation}
Combining \eqnreftwo{eq:02_Acoustical_Theory:PW_Quadrature_Binaural}{eq:02_Acoustical_Theory:A2mu_Pinv}, we obtain
\begin{equation}\label{eq:02_Acoustical_Theory:PW_Pinv_Binaural_Matrix}
p^{\text{L,R}}(t) = \left( \mathbf{h}^{\text{L,R}} \ast \left( \mathbf{Y}^{+} \cdot \mathbf{a} \right) \right) (t).
\end{equation}
However, this results in identical binaural signals (assuming the same grid of $Q$ directions) between \eqnreftwo{eq:02_Acoustical_Theory:VA_Binaural_Matrix}{eq:02_Acoustical_Theory:PW_Pinv_Binaural_Matrix}, since we have effectively set $w_q \mu(t,\hat{v}_q) = g_q(t)$.

\subsection{Spherical-harmonic HRTFs}\label{sec:02_Acoustical_Theory:SH_Binaural}
A more computationally efficient approach to rendering ambisonics to binaural is to filter each ambisonics signal by the corresponding term in a spherical-harmonic decomposition of an HRTF.
The spherical-harmonic HRTFs can be computed via pseudoinverse, as given by \citet[section IV]{RafaelyAvni2010}, such that
\begin{equation}\label{eq:02_Acoustical_Theory:SH_HRTFs}
\mathbf{\tilde{h}}^{\text{L,R}}(t) = 
\begin{bmatrix}
\tilde{h}_{0}^{\text{L,R}}(t) & \tilde{h}_{1}^{\text{L,R}}(t) & \cdots & \tilde{h}_{N-1}^{\text{L,R}}(t)
\end{bmatrix}
 = \mathbf{h}^{\text{L,R}}(t) \cdot \mathbf{Y}^{+}.
\end{equation}
The resulting binaural pressure signals are then given by
\begin{equation}\label{eq:02_Acoustical_Theory:SH_Binaural}
p^{\text{L,R}}(t) \approx \sum_{n=0}^{N-1} \left( \tilde{h}_n^{\text{L,R}} \ast a_n \right) (t)
 = \left( \mathbf{\tilde{h}}^{\text{L,R}} \ast \mathbf{a} \right) (t).
\end{equation}
Combining~\eqnreftwo{eq:02_Acoustical_Theory:SH_HRTFs}{eq:02_Acoustical_Theory:SH_Binaural}, we obtain
\begin{equation}\label{eq:02_Acoustical_Theory:SH_Binaural_Matrix}
p^{\text{L,R}}(t) = \left( \left( \mathbf{h}^{\text{L,R}} \cdot \mathbf{Y}^{+} \right) \ast \mathbf{a} \right) (t),
\end{equation}
which is identical (again, assuming the same grid of $Q$ directions) to both \eqnreftwo{eq:02_Acoustical_Theory:VA_Binaural_Matrix}{eq:02_Acoustical_Theory:PW_Pinv_Binaural_Matrix}.