In this chapter, we proposed and characterized an interpolation-based method for virtual navigation, wherein the subset of microphones to be used is parametrically determined to ensure that the region of validity restriction is not violated.
An existing alternative method, in which navigation is performed by computing a weighted average of the ambisonics signals from each microphone, was shown in \secref{sec:08_Proposed_Method:Fundamental_Problems} to incur spectral coloration due to comb-filtering and localization errors due to the precedence effect.
The proposed method, described in \secref{sec:08_Proposed_Method:Proposed_Techniques}, employs knowledge of the locations of any near-field sources in order to determine which ambisonics microphones are valid for use in the navigation calculation as a function of the desired listening position.
Additionally, at low frequencies, the proposed method applies a matrix of regularized least-squares inverse filters to estimate the ambisonics signals at the listening position, while at high frequencies, the weighted average method is employed.

Following the numerical simulation framework laid out in \chapref{chap:06_Simulation_Framework}, we compared the proposed method to the weighted average method.
These two methods were evaluated for a linear array geometry (illustrated in \figref{fig:06_Simulation_Framework:Linear_Geometry}) in terms of the metrics enumerated in \secref{sec:06_Simulation_Framework:Metrics} for sound level, spectral coloration, source localization, and diffuseness.
Results show that, for interior sources (as defined in \figref{fig:06_Simulation_Framework:Linear_Geometry}), both methods fail to achieve a sufficient reproduced sound level.
As sound level is well-known to be a primary distance cue \citep[section 3.1.1]{Zahorik2005}, the impact of these errors on distance perception should be investigated through subjective listening tests.

Otherwise, for interior sources, the proposed method achieves a significant improvement (in terms of coloration, localization, and diffuseness) over the existing method.
In particular, the proposed method yields significantly improved localization errors over the existing method for large microphone spacings (larger than $0.5$~m).
This improvement is primarily a result of excluding the invalid microphone, which would otherwise add spectral coloration, corrupt the localization information, and increase diffuseness in the reproduced signals.
Additionally, for small microphone spacings (smaller than $0.5$~m) and exterior sources, the proposed method achieves slightly smaller spectral errors than does the existing method.
This is due to the widening (as microphone spacing decreases) of the frequency range over which the regularized least-squares interpolation filters achieve a nearly flat frequency response (see \figref{fig:08_Proposed_Method:Azimuth_Dependence}). % (cf.~\citet[Fig.~4b]{TylkaChoueiri2016}).
% this is basically the main result of the sourceAz section

Results also show that the performance of the proposed method is largely independent of the input ambisonics order (see \secref{sec:08_Proposed_Method:Results}).
As this is primarily a consequence of our order-independent choice of critical frequency for the hybrid interpolation filters (see \eqnref{eq:08_Proposed_Method:Hybrid_XO_Freq}), future refinements to the proposed method should explore the use of order-dependent critical frequencies in an effort to better leverage the additional information about the sound field contained in the higher-order signals.
Ideally, this information could be used to further improve localization accuracy for interior sources and/or mitigate the spectral coloration induced by the proposed method for exterior sources.

The results presented in this chapter further corroborate the primary finding of \chapref{chap:07_Characterization_Extrapolation}: that violating the region of validity restriction introduces significant errors (in this case, primarily in terms of coloration and localization).
As demonstrated here, our proposed method successfully mitigates these errors by parametrically ensuring that the region of validity restriction is not violated as the listener navigates.
In \chapref{chap:09_Thiergart_Comparison}, we consider another parametric interpolation method, which is based on a time-frequency analysis of the sound field, and compare its performance to that of our proposed method.