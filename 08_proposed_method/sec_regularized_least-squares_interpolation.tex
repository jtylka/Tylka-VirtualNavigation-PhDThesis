To compute the interpolation filters, we modify the pseudoinverse-based interpolation method described in \secref{sec:03_Navigation_Techniques:Pinv_Technique} and propose a particular regularization function.
First, we modify \eqnref{eq:03_Navigation_Techniques:Linear_System_Matrices}, such that the matrices are given by
\begin{equation}\label{eq:08_Proposed_Method:Linear_System_Matrices}
\mathbf{M} = 
    \left[ \begin{array}{c}
    \sqrt{w_1} \left( \mathbf{T}(-\vec{r}_1) \right)^\text{T} \\
    \sqrt{w_2} \left( \mathbf{T}(-\vec{r}_2) \right)^\text{T} \\
    \vdots\\
    \sqrt{w_P} \left( \mathbf{T}(-\vec{r}_P) \right)^\text{T}
    \end{array} \right]
,\quad
\mathbf{y} = 
    \left[ \begin{array}{c}
    \sqrt{w_1} \mathbf{b}_1\\
    \sqrt{w_2} \mathbf{b}_2\\
    \vdots\\
    \sqrt{w_P} \mathbf{b}_P
    \end{array} \right],
\end{equation}
where $w_p$ is the interpolation weight for the $p^\textrm{th}$ microphone and recall that $\vec{r}_p = \vec{r}_0 - \vec{u}_p$ is the position of the listener relative to the $p^\textrm{th}$ microphone.

Next, we compute the singular value decomposition of $\mathbf{M}$, such that $\mathbf{M} = \mathbf{U} \Sigma \mathbf{V}^*$, where $(\cdot)^*$ represents conjugate-transposition.
This allows us to compute a regularized pseudoinverse of $\mathbf{M}$, given by \citep[section 5.1]{Hansen1998}
\begin{equation}
\mathbf{L} = \mathbf{V} \Theta \Sigma^{+} \mathbf{U}^*,
\end{equation}
where $(\cdot)^+$ represents pseudoinversion, and $\Theta$ is a square, diagonal matrix whose elements are given by
\begin{equation}
\Theta_{nn} = \frac{\sigma_n^2}{\sigma_n^2 + \beta},
\end{equation}
where $\sigma_n$ is the $n^\textrm{th}$ singular value of $\mathbf{M}$, with $n \in [1,N_\textrm{max}]$.
(Recall that $N_\textrm{max} = (L_\textrm{max} + 1)^2$ and $L_\textrm{max}$ is defined in \eqnref{eq:03_Navigation_Techniques:Pinv_Interpolation_Lmax}.)
In general, the regularization parameter $\beta$ may be a function of frequency.
Here, we choose the magnitude of a high-shelf filter as the regularization function, given by \citep[section 5.2]{Zolzer2008}
\begin{equation}
\beta(k) = \beta_0 \left| \frac{G_{\pi} i \frac{k}{k_0} + 1}{i \frac{k}{k_0} + G_{\pi}} \right|,
\end{equation}
where $G_{\pi}$ is the amplitude of the high-shelf filter and $k_0$ is its zero-dB crossing, which, for convenience, we take to be the same as that given below in \eqnref{eq:08_Proposed_Method:Hybrid_XO_Freq}.
(In our original publication, we had chosen $k_0 = 1 / \Delta$ for simplicity \citep[Eq.~(17)]{TylkaChoueiri2016}, but we did not consider cases where $P \neq 2$.)
We then let
\begin{equation}
\beta_0 = \frac{1}{\sigma_0} \max_n \sigma_n,
\end{equation}
with some constant $\sigma_0 \gg 1$.
Note that the singular values ($\sigma_n$) of $\mathbf{M}$ are calculated for each frequency, so, in general, $\beta_0$ is also frequency-dependent.
Here, we choose $G_{\pi} = 10^{1.5}$ (i.e., 30~dB) and $\sigma_0 = 1000$.

Finally, we obtain an estimate of $\mathbf{a}$ at each $k$, given by
\begin{equation}
\mathbf{\tilde{a}}(k) = \mathbf{L}(k) \cdot \mathbf{y}(k).
\end{equation}
Note that, as with the weighted average method, we may choose to drop the higher-order terms in $\mathbf{\tilde{a}}$ such that we keep only up to order $L_\textrm{out}$, where $L_\textrm{out} \leq L_\textrm{max}$.

Also note that we can factor out the interpolation weights into a diagonal matrix, such that
\begin{equation}
\mathbf{\tilde{a}}(k) = \left( \mathbf{L}(k) \cdot
    \left[ \begin{array}{c c c c}
    \sqrt{w_1}\, \mathbf{I} & \mathbf{0} & \cdots & \mathbf{0}\\
    \mathbf{0} & \sqrt{w_2}\, \mathbf{I} & \ddots & \vdots\\
    \vdots & \ddots & \ddots & \mathbf{0}\\
    \mathbf{0} & \cdots & \mathbf{0} & \sqrt{w_P}\, \mathbf{I}
    \end{array} \right]
 \right) \cdot
     \left[ \begin{array}{c}
    \mathbf{b}_1(k)\\
    \mathbf{b}_2(k)\\
    \vdots\\
    \mathbf{b}_P(k)
    \end{array} \right],
\end{equation}
where $\mathbf{0}$ is an $N_\textrm{in} \times N_\textrm{in}$ matrix of zeros, and $\mathbf{I}$ is the $N_\textrm{in} \times N_\textrm{in}$ identity matrix.
For compactness, we let
\begin{equation}
\mathbf{L}_w(k) = \mathbf{L}(k) \cdot
    \left[ \begin{array}{c c c c}
    \sqrt{w_1}\, \mathbf{I} & \mathbf{0} & \cdots & \mathbf{0}\\
    \mathbf{0} & \sqrt{w_2}\, \mathbf{I} & \ddots & \vdots\\
    \vdots & \ddots & \ddots & \mathbf{0}\\
    \mathbf{0} & \cdots & \mathbf{0} & \sqrt{w_P}\, \mathbf{I}
    \end{array} \right].
\end{equation}