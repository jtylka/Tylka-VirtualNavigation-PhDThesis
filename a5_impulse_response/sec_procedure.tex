Given an input signal $x(t)$ and an IR $h(t)$, the system's output is given by
\begin{equation}
w(t) = (x \ast h) (t),
\end{equation}
where $\ast$ denotes convolution.
The recorded signal $y(t)$, however, includes noise $n(t)$ (as well as any non-linear distortion terms), and is given by
\begin{equation}
y(t) = w(t) + n(t).
\end{equation}
A block digram of the system under test as well as these various signals is given in \figref{fig:A5_Impulse_Response:Procedure:Block_Diagram}.

\begin{figure}[t]
\centering
\begin{tikzpicture}[scale=1]

\def\sumradius{0.25};
\def\spacing{1};
\def\blockwidth{3};
\def\blockheight{1.5};

\draw[thick,->] (-\spacing,0) node[left]{$x$} -- (0,0);
\draw (0,-\blockheight/2) rectangle (\blockwidth,\blockheight/2) node[pos=0.5]{$h$};
\draw[thick,->] (\blockwidth,0) -- (\blockwidth+\spacing,0) node[above, pos=0.5]{$w$};
\draw[thick,->] (\blockwidth+\spacing+\sumradius,\spacing+\sumradius) node[above]{$n$} -- (\blockwidth+\spacing+\sumradius,\sumradius);
\draw (\blockwidth+\spacing+\sumradius,0) circle (\sumradius cm) node[]{$+$};
\draw[thick,->] (\blockwidth+\spacing+2*\sumradius,0) -- (\blockwidth+2*\spacing+2*\sumradius,0) node[right]{$y$};
\end{tikzpicture}
\caption{Block diagram of a system under test.}
\label{fig:A5_Impulse_Response:Procedure:Block_Diagram}
\end{figure}

The IR of an acoustical system is obtained by deconvolving the recorded output signal $y$ by the known input signal $x$.
In practice, deconvolution is typically performed by dividing the corresponding spectra in the frequency domain via the fast Fourier transform (FFT).
This is equivalent to convolving the recorded signal with the input signal's exact inverse, whose frequency spectrum is equal to the reciprocal of the input spectrum.
We shall refer to this procedure as \textit{exact deconvolution} of the recorded signal.
Ideally, if $n(t) = 0$, the system's impulse response would be computed exactly by
\begin{equation}
h(t) = \mathcal{F}^{-1} \left[ \frac{\mathcal{F} \left[ w(t) \right]}{\mathcal{F} \left[ x(t) \right]} \right].
\end{equation}
However, since we can only measure $y(t)$, we instead obtain an estimate of $h(t)$, given by
\begin{equation}
h(t) \approx \mathcal{F}^{-1} \left[ \frac{\mathcal{F} \left[ y(t) \right]}{\mathcal{F} \left[ x(t) \right]} \right].
\end{equation}

%The SNR spectrum as a function of frequency is defined as
%\begin{equation}
%\text{SNR}(\omega) \equiv 10 \log_{10} \frac{\left| W(\omega) \right|^2}{\left| N(\omega) \right|^2},
%\end{equation}
%where $W(\omega)$ is the Fourier transform of $w(t)$, and similarly for $N(\omega)$ and $n(t)$.
%Also, the total SNR is defined as
%\begin{equation}
%\text{SNR} \equiv 10 \log_{10} \frac{\displaystyle \int_{-\infty}^{\infty} \left| W(\omega) \right|^2 d\omega}{\displaystyle \int_{-\infty}^{\infty} \left| N(\omega) \right|^2 d\omega}.
%\end{equation}