Implemented in this toolkit are several basic methods of decoding ambisonics (described below),
but the Ambisonic Decoder Toolbox (ADT) by \citet{Heller2012} is a much more comprehensive tool for creating state-of-the-art ambisonic decoders (available online).\citefooturl{ADTURL}
Consequently, one intended use of this toolkit is to add custom HRTFs to an existing ambiX decoder preset (such as those generated using the ADT).%
\footnote{At present, the SABRE toolkit is only compatible with single-band decoders.
Consequently, multi-band decoders should be implemented in parallel (as a bank of single-band decoders), downstream of a crossover network.}
Note that doing so requires the user to specify the grid of speaker positions, as that information is not explicitly contained in ambiX decoder presets.

Generally, the ambiX binaural decoder employs the virtual ambisonics rendering approach, as described in \secref{sec:02_Acoustical_Theory:VA_Binaural}.
However, with this toolkit, we take advantage of that architecture (a decoding matrix followed by HRTF filters) and the linearity of the processing chain, in order to also implement the plane-wave rendering approach (as described in \secref{sec:02_Acoustical_Theory:PW_Quadrature_Binaural}) and the spherical-harmonic HRTF rendering approach (as described in \secref{sec:02_Acoustical_Theory:SH_Binaural}).
For each of the decoding methods described below, we seek an equation of the form of \eqnref{eq:02_Acoustical_Theory:VA_Binaural_Matrix}, such that the binaural pressure signals, $p^{\text{L,R}}$, are given by
\begin{equation}\tag{\ref{eq:02_Acoustical_Theory:VA_Binaural_Matrix}}
p^{\text{L,R}}(t) = \left( \mathbf{h}^{\text{L,R}} \ast \left( \mathbf{D} \cdot \mathbf{a} \right) \right)(t).
\end{equation}

\subsection{Basic decoding}
In this toolkit, we have implemented basic ambisonics decoding functionality in order to compute,
for a given loudspeaker array configuration, the basic (pseudoinverse) decoder.
Recalling~\eqnref{eq:02_Acoustical_Theory:PinvDecoder}, this decoder matrix $\mathbf{D}$ is given by
\begin{equation}\tag{\ref{eq:02_Acoustical_Theory:PinvDecoder}}
\mathbf{D} = \mathbf{Y}^{+}.
\end{equation}
The output signals from this decoder matrix are then filtered by the vector, $\mathbf{h}^{\text{L,R}}$, of HRIRs for a given listener and for the directions of each loudspeaker.

\subsection{Quadrature decoding}
For specific spherical grids (e.g., those derived by \citet{FliegeMaier1999}) that have known corresponding quadrature weights,
we can alternatively compute a quadrature decoder.
Recall that, in the case of a finite-term plane-wave expansion (computed on this spherical grid) of a sound field, the resulting binaural pressure signals are given by
\begin{equation}\tag{\ref{eq:02_Acoustical_Theory:PW_Quadrature_Binaural_Matrix}}
p^{\text{L,R}}(t) = \left( \mathbf{h}^{\text{L,R}} \ast \left( \mathbf{W} \cdot \mathbf{Y}^{\textrm{T}} \cdot \mathbf{F}^{-1} \cdot \mathbf{a} \right) \right) (t),
\end{equation}
where, as defined in \secref{sec:02_Acoustical_Theory:Binaural_Rendering}, we have let
\begin{equation}\tag{\ref{eq:02_Acoustical_Theory:PW_Quadrature_Weights}}
\mathbf{W} = \text{diag} \left\{ \begin{bmatrix} w_1 & w_2 & \cdots & w_Q \end{bmatrix} \right\}.
\end{equation}
Comparing \eqnreftwo{eq:02_Acoustical_Theory:VA_Binaural_Matrix}{eq:02_Acoustical_Theory:PW_Quadrature_Binaural_Matrix}, we obtain a decoder matrix given by
\begin{equation}
\mathbf{D} = \mathbf{W} \cdot \mathbf{Y}^{\textrm{T}} \cdot \mathbf{F}^{-1},
\end{equation}
whose output signals would again be filtered by the same vector, $\mathbf{h}^{\text{L,R}}$, of HRIRs.

\subsection{Compact decoding}
In order to make the decoding matrix as compact as possible, we compute spherical harmonic HRTFs, as described in \secref{sec:02_Acoustical_Theory:SH_Binaural}.
More generally, we can compute the spherical harmonic HRTFs by performing the matrix multiplication with any decoder matrix, i.e., $\mathbf{\tilde{h}}^{\text{L,R}}(t) = \mathbf{h}^{\text{L,R}}(t) \cdot \mathbf{D}$.
This process simply combines the decoding matrix and per-direction HRIRs into a single set of filters.\footnote{Conceptually,
each compacted HRIR $\tilde{h}_{n}^{\text{L,R}}(t)$ represents the signals at the ears in response to an impulse sent through the $n^{\textrm{th}}$ ambisonics channel.}
Thus, we can write the ambisonics decoder as simply an $N \times N$ identity matrix, i.e., $\mathbf{\tilde{D}} = \mathbf{I}_{(N \times N)}$, such that $\mathbf{\tilde{h}}^{\text{L,R}}(t) \cdot \mathbf{\tilde{D}} = \mathbf{h}^{\text{L,R}}(t) \cdot \mathbf{D}$.

\subsection{Normalization}
For each HRIR (compacted or not), we first compute the maximum gain, $\alpha_q$, across the entire frequency response.
Subsequently, we attenuate each HRIR and amplify the corresponding row of the decoder matrix by that gain, i.e.,
\begin{equation}
\mathbf{\hat{h}}^{\text{L,R}}(t) = \mathbf{h}^{\text{L,R}}(t) \cdot \mathbf{G}^{-1},
\quad\quad
\text{and}
\quad\quad
\mathbf{\hat{D}} = \mathbf{G} \cdot \mathbf{D},
\end{equation}
where
\begin{equation}
\mathbf{G} = \text{diag} \left\{ \begin{bmatrix} \alpha_1 & \alpha_2 & \cdots & \alpha_Q \end{bmatrix} \right\},
\end{equation}
such that $\mathbf{\hat{h}}^{\text{L,R}}(t) \cdot \mathbf{\hat{D}} = \mathbf{h}^{\text{L,R}}(t) \cdot \mathbf{D}$.
Finally, we normalize the overall decoder matrix such that the maximum absolute value of any element in matrix is unity.