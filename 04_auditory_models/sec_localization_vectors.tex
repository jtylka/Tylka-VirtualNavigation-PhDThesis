To predict the localization of sound in multichannel playback systems, \citet{Gerzon1992} defines two localization metrics: the velocity and energy vectors.
Note that these definitions assume that the loudspeakers are equidistant from the listening position, such that the signals all arrive coincidentally.

The velocity vector is used to predict localization due to interaural time differences at low frequencies ($<700$~Hz) and is given by \citep{Gerzon1992}
\begin{equation}\label{eq:04_Auditory_Models:Velocity_Vector}
\vec{\nu}_{\text{V}}(f) = \textrm{Re} \left\{ \frac{ \sum_q G_q(f) \hat{v}_q}{ \sum_q G_q(f)} \right\},
\end{equation}
where $\text{Re} \left\{ \cdot \right\}$ denotes taking the real part of the argument, $G_q$ is the complex-valued, frequency-dependent ``gain'' of the $q^{\text{th}}$ source, and $\vec{v}_q$ is the position of that source.
The energy vector is used to predict localization due to interaural level differences at higher frequencies (500~Hz -- 5~kHz) and is given by \citep{Gerzon1992}
\begin{equation}\label{eq:04_Auditory_Models:Energy_Vector}
\vec{\nu}_{\text{E}}(f) = \frac{ \sum_q |G_q(f)|^2 \hat{v}_q}{ \sum_q |G_q(f)|^2}.
\end{equation}
The directions of these vectors indicate the expected localization direction and their magnitudes indicate the quality of the localization.
Ideally, the vectors should have a magnitude equal to unity and point in the direction of the virtual source.