Generally, we use the term \textit{source localization} to refer to the perceived direction of a sound source relative to a listener.
Here, we review two well-established primitive localization vectors and describe two more sophisticated localization models.
For a predicted localization direction $\hat{\nu}$ (computed at the position of the listener), the localization error $e_\nu$ is given by
\begin{equation}\label{eq:04_Auditory_Models:Localization_Error}
e_\nu = \cos^{-1} \left( \hat{\nu} \cdot \hat{s}_0{}' \right),
\end{equation}
where $\hat{s}_0{}'$ is the direction of the source relative to the listener, found by normalizing the vector $\vec{s}_0{}' = \vec{s}_0 - \vec{r}_0$.

%% Gerzon's Localization Vectors
\subsection{Velocity and energy vectors (\texorpdfstring{$\vec{\nu}_{\text{V}}, \vec{\nu}_{\text{E}}$}{vV, vE})}\label{sec:04_Auditory_Models:Localization_Vectors}
To predict the localization of sound in multichannel playback systems, \citet{Gerzon1992} defines two localization metrics: the velocity and energy vectors.
Note that these definitions assume that the loudspeakers are equidistant from the listening position, such that the signals all arrive coincidentally.

The velocity vector is used to predict localization due to interaural time differences at low frequencies ($<700$~Hz) and is given by \citep{Gerzon1992}
\begin{equation}\label{eq:04_Auditory_Models:Velocity_Vector}
\vec{\nu}_{\text{V}}(f) = \textrm{Re} \left\{ \frac{ \sum_q G_q(f) \hat{v}_q}{ \sum_q G_q(f)} \right\},
\end{equation}
where $\text{Re} \left\{ \cdot \right\}$ denotes taking the real part of the argument, $G_q$ is the complex-valued, frequency-dependent ``gain'' of the $q^{\text{th}}$ source, and $\vec{v}_q$ is the position of that source.
The energy vector is used to predict localization due to interaural level differences at higher frequencies (500~Hz -- 5~kHz) and is given by \citep{Gerzon1992}
\begin{equation}\label{eq:04_Auditory_Models:Energy_Vector}
\vec{\nu}_{\text{E}}(f) = \frac{ \sum_q |G_q(f)|^2 \hat{v}_q}{ \sum_q |G_q(f)|^2}.
\end{equation}
The directions of these vectors indicate the expected localization direction and their magnitudes indicate the quality of the localization.
Ideally, the vectors should have a magnitude equal to unity and point in the direction of the virtual source.

%% Stitt's Localization Model
\subsection{Precedence-effect-based energy vector (\texorpdfstring{$\vec{\nu}_{\text{PE}}$}{vPE})}\label{sec:04_Auditory_Models:PE_Energy_Vector}
Recently, \citet{Stitt2016} proposed an extension to incorporate the precedence effect into the energy vector of \citet{Gerzon1992}.
In their paper, \citeauthor{Stitt2016} also showed that their proposed model achieves improved localization prediction accuracy compared to the binaural models of \citet{Dietz2011} and \citet{Lindemann1986a}.
The model predicts localization by computing a weighted average of position vectors for each of a finite set of point sources.
The model assumes that each source is generating the same signal, although perhaps at different amplitudes, and perhaps with different distances between each source and the listener.
Perceptual weights are first computed for each source based its time-of-arrival to and amplitude at the listening position.
The calculation of these weights is not trivial, but briefly, it involves weighting earlier signals more heavily, but adjusting the weights of later signals based on their amplitudes and directions-of-arrival (see the original paper for details).

The predicted localization is then given as a vector by
\begin{equation}\label{eq:PE_Energy_Vector}
\vec{\nu}_{\text{PE}} = \frac{ \displaystyle \sum_{q=1}^Q |w_q(\alpha) G_q / v_q|^2 \, \hat{v}_q}{ \displaystyle \sum_{q=1}^Q |w_q(\alpha) G_q / v_q|^2},
\end{equation}
where $w_q$ is the perceptual weight for the $q^\text{th}$ source, $G_q$ is the amplitude of the $q^\text{th}$ source, $v_q$ is the distance of the $q^\text{th}$ source, $\hat{v}_q$ is a unit vector pointing towards the $q^\text{th}$ source, and $\alpha \in [0,1]$ is a free parameter which specifies the relative importance of stationary (i.e., time-averaged) to transient information in the stimulus signal.%
\footnote{For example, a highly transient signal is expected to require a low value of $\alpha$, while a more stationary signal would require a higher value \citep{Stitt2016,Stitt2017}.}

%% Dietz' Model
\subsection{Binaural localization model (\texorpdfstring{$\vec{\nu}_{\text{B}}$}{vB})}\label{sec:04_Auditory_Models:Binaural_Localization_Model}
We also consider the binaural localization model of \citet{Dietz2011}.
In order to compute the required binaural impulse responses, the ambisonics impulse response is first converted to plane-wave impulse responses using \eqnref{eq:02_Acoustical_Theory:A2mu}.
The binaural impulse responses are then computed by \eqnref{eq:02_Acoustical_Theory:PW_Quadrature_Binaural}.

An implementation of this model is freely available in the auditory modeling toolbox,\citefooturl{AMTURL} authored by \citet{SondergaardMajdak2013}.
The core of the model \citep{Dietz2011} emulates inner-ear processes (e.g., auditory band filtering, signal compression, half-wave rectification, etc.) in order to compute binaural parameters such as the interaural time difference (ITD), interaural level difference, and interaural coherence in a range of frequency bands for both the envelope and fine-structure components of the binaural signals (see the original paper for details).
In this work, we adopted the extension proposed by \citet{Wierstorf2013}, in which an ITD-to-azimuth lookup table is first generated for each subject using that subject's measured HRTFs on the frontal horizontal plane.
The stimulus signal is then filtered by the binaural impulse responses and the ITD is computed using the original model.
The model yields ITD in a set of frequency bands, which are then converted to azimuth via the lookup table.
Outliers beyond $30^\circ$ away from the median azimuth are then removed, and finally, a single predicted azimuth, $\phi_{\text{B}}$, is computed as the weighted average over frequency, with weights given by the rms signal amplitude in each frequency band.
The predicted localization direction is then given by
\begin{equation}
\vec{\nu}_{\text{B}} = (\cos \phi_{\text{B}}, \sin \phi_{\text{B}}, 0).
\end{equation}