Generally, we use the term \textit{spectral coloration} to refer to any changes in the spectral content of a signal, for instance due to the application of a filter whose magnitude response is not constant.
However, as not all colorations are equally audible, metrics which mimic auditory filtering processes offer more perceptually-relevant estimates of colorations.
Here, we review five metrics that all aim to quantify such colorations.
For several of these metrics, we further define the \textit{spectral range}, given by
\begin{equation}\label{eq:Spectral_Range}
\rho_S = \max_c S(f_c) - \min_c S(f_c),
\end{equation}
and the \textit{spectral deviation}, given by
\begin{equation}\label{eq:Spectral_Deviation}
\sigma_S = \sqrt{\frac{1}{N_b} \sum_{c = 1}^{N_b} \left( S(f_c) - \overline{S} \right)^2},
\end{equation}
where $S$ is some metric (specified in dB, unless stated otherwise) and $\overline{S}$ is its average over all frequency bands.
For these metrics, we use the frequency range $f \in [f_\text{L}, f_\text{U}]$ with $f_\text{L} = 50$~Hz and $f_\text{U} = 21$~kHz, as recommended by~\citet{Boren2015}.

%% Scharer's ABSE
\subsection{Auditory band spectral error (\texorpdfstring{$\eta$}{eta})}\label{sec:04_Auditory_Models:Coloration_Metrics:ABSE}
The \textit{auditory band spectral error} (ABSE), adapted from \citet[Eq.~(9)]{ScharerLindau2009}, is given by
\begin{equation}\label{eq:ABSE}
\eta(f_c) = 10 \log_{10} \left( \frac{\int |H_\Gamma(f;f_c)| |\tilde{A}_0(f)|^2 df}{\int |H_\Gamma(f;f_c)| |A_0(f)|^2 df} \right),
\end{equation}
where $A_0$ and $\tilde{A}_0$ are the zeroth-order terms of the reference and test (respectively) ambisonics transfer functions and integration is taken over all frequencies.
For this and other coloration metrics requiring an auditory filter bank, we use ERB-spaced center frequencies \citep{GlasbergMoore1990} spanning the range $f \in [f_\text{L}, f_\text{U}]$ and denoted $f_c$ for $c \in [1, N_b]$.
In this case, we define the spectral range and deviation of the ABSE: $\rho_\eta, \sigma_\eta$, respectively.

 %% Boren's Epk and En
\subsection{Peak and notch errors (\texorpdfstring{$E_{\text{pk}}, E_\text{n}$}{Epk, En})}\label{sec:04_Auditory_Models:Coloration_Metrics:Epk}
The \textit{peak and notch errors} ($E_{\text{pk}}, E_\text{n}$) were defined by \citet{Boren2015}
and essentially quantify the average peak (or notch) height (depth) in a frequency response over a certain frequency range.
First, the difference (in dB) is computed between finely- and coarsely-smoothed versions of the the normalized free-field transfer function $\tilde{A}_0(f)/A_0(f)$, i.e.,
\begin{equation}
D(f) = 20 \log_{10} \left( \frac{ \mathcal{S}\left( \left| \tilde{A}_0(f)/A_0(f) \right| ; 1/48 \right) }{ \mathcal{S}\left( \left| \tilde{A}_0(f)/A_0(f) \right| ; 1 \right) } \right),
\end{equation}
where $\mathcal{S}(X; B)$ denotes fractional-octave smoothing applied to the spectrum $X$ with smoothing bandwidth $B$ octaves.
Here, we used the fractional-octave smoothing method of \citet{Tylka2017}, reproduced in \apxref{chap:A3_Smoothing_Weights}.

The peak- and notch-finding algorithms described by \citeauthor{Boren2015} are then applied to find the frequencies $f_1^\uparrow, f_2^\uparrow, \dots f_{N_\text{pk}}^\uparrow$ of all $N_\text{pk}$ spectral peaks and $f_1^\downarrow, f_2^\downarrow, \dots f_{N_\text{n}}^\downarrow$  of all $N_\text{n}$ spectral notches in the range $f \in [f_\text{L}, f_\text{U}]$.
The peak and notch errors are then given by \citep[Eq.~(1)]{Boren2015}
\begin{equation}
E_\text{pk} = \frac{\sum_{j = 1}^{N_\text{pk}} D(f_j^\uparrow)}{3 \log_2 (f_\text{U}/f_\text{L})}
\quad\quad\text{and}\quad\quad
E_\text{n} = \frac{\sum_{j = 1}^{N_\text{n}} (-D(f_j^\downarrow))}{3 \log_2 (f_\text{U}/f_\text{L})},
\end{equation}
respectively.
Note that, since $D$ is given in dB, the negative sign in the second equation typically ensures that both metrics are positive-valued.

%% Kates' CS
\subsection{Central spectrum (CS)}
The \textit{central spectrum} (CS) was defined by \citet{Kates1984} for use as a metric for comparing loudspeaker responses.
Consequently, it may be employed using only the free-field transfer functions of the test and reference signals.
Essentially, the metric is computed as the sum of the Fourier coefficients and critical band energies of a time-weighted autocorrelation of the input signal (see the original paper for details).
We then compute the difference (in dB) between the central spectra for the test and reference signals, given by
\begin{equation}
e_\text{CS}(f_c) = \widetilde{\text{CS}}(f_c) - \text{CS}(f_c).
\end{equation}
As done for the ABSE, we define the spectral range and deviation of the CS: $\rho_{e_\text{CS}}, \sigma_{e_\text{CS}}$, respectively.

%% Pulkki's CLL
\subsection{Composite loudness level (CLL)}\label{sec:04_Auditory_Models:Coloration_Metrics:Pulkki_CLL}
The \textit{composite loudness level} (CLL) spectrum was defined by \citet[section 1.1]{Pulkki1999} to give an estimate of perceived loudness.
Computing the CLL requires binaural impulse responses, which we compute using \eqnref{eq:02_Acoustical_Theory:PW_Quadrature_Binaural}.
The CLL spectrum is then computed as the sum of the two ears' loudness level spectra, each of which are given in phons%
\footnote{The \textit{phon} is a unit of perceived loudness, and is related to dB SPL via the equal-loudness contours originally described by \citet{FletcherMunson1933}.
Similar to dB, a change in perceived loudness is computed via a difference of values in phons.}
and found via a gammatone filter bank, half-wave rectification, low-pass filtering, and time-averaging of the signal energy in each band (see the original paper for details).
We then compute the difference (in phons) between the CLL spectra for the test and reference samples, given by
\begin{equation}\label{eq:04_Auditory_Models:Pulkki_CLL}
e_\text{CLL}(f_c) = \widetilde{\text{CLL}}(f_c) - \text{CLL}(f_c).
\end{equation}
We then define the spectral range and deviation of the CLL: $\rho_{e_\text{CLL}}, \sigma_{e_\text{CLL}}$, respectively.

%% Wittek's SA
\subsection{Internal spectrum (IS)}\label{sec:04_Auditory_Models:Coloration_Metrics:Wittek_IS}
\citet{Wittek2007} adapted the \textit{internal spectrum} (IS) defined by \citet[chapter 5]{Salomons1995PhD} in order to define so-called \textit{spectral alterations}.
These spectral alterations are computed as the difference (in dB) between the internal spectra for the test and reference signals, given by
\begin{equation}
e_\text{IS}(f_c) = \widetilde{\text{IS}}(f_c) - \text{IS}(f_c).
\end{equation}
According to \citet[section 3.2.5]{Wittek2007}, the IS for each signal is given as the average of the binaural power spectra, i.e.,
\begin{equation}
\text{IS}(f_c) = 10 \log_{10} \left( \frac{P^\text{L}(f_c) + P^\text{R}(f_c)}{2} \right).
\end{equation}
Here, $P^{\text{L},\text{R}}$ are the binaural power spectra after critical-band filtering, given by \citep[Eq.~(5.12)]{Salomons1995PhD}
\begin{equation}
P^{\text{L},\text{R}}(f_c) = \frac{\displaystyle \int_{-\infty}^\infty |H_\text{P}(f;f_c)| |\psi^{\text{L},\text{R}}(f)|^2 df}{\displaystyle \int_{-\infty}^\infty |H_\text{P}(f;f_c)| df},
\end{equation}
where $H_\text{P}(f;f_c)$ are Patterson's auditory filters (which are very similar in magnitude response to gammatone filters, as shown in \figref{fig:04_Auditory_Models:Auditory_Filters}) as specified by \citet[Eq.~(5.9)]{Salomons1995PhD} and $\psi^{\text{L},\text{R}}$ are the binaural transfer functions, given by the Fourier transform of the binaural impulse responses from \eqnref{eq:02_Acoustical_Theory:PW_Quadrature_Binaural}.
Again, we define the spectral range and deviation of the IS: $\rho_{e_\text{IS}}, \sigma_{e_\text{IS}}$, respectively.
Note that $\rho_{e_\text{IS}}$ is precisely equivalent to the $A_0$-measure defined by \citet[section 3.2.6]{Wittek2007}, which is based on the $A_0$-criterion defined by \citet[section 5.4]{Salomons1995PhD}.
Additionally, $\sigma_{e_\text{IS}}$ is essentially equivalent to the ``spectral deviation'' described by \citet[section 3.2.6]{Wittek2007}.