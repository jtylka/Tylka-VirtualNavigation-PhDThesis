In this work we explored the ability (or lack thereof) of fractional-octave smoothing methods to preserve the log-frequency symmetry of frequency spectra.
Specifically, we examined two existing methods of smoothing, the first of which uses a symmetric (on a linear frequency scale) window of the correct bandwidth, but whose cutoff frequencies are not equidistant from the center frequency when viewed on a log-frequency scale, and therefore do not correspond to the \textit{correct} fractional-octave band.
The second method requires that the raw spectrum first be interpolated to a log-frequency scale, a process that necessarily introduces errors, but simplifies the fractional-octave smoothing procedure to a moving-average operation with a symmetric window that corresponds well with the correct fractional-octave band.
We proposed a third smoothing method that is able to accurately replicate the smoothed spectrum of the second method but without the need for interpolation.
This method is fully compatible with other FFT-based smoothing techniques such as complex smoothing (see \citet{HatziantoniouMourjopoulos2000}) and can be employed with any smoothing window (e.g., Hanning window, band-pass filter, etc.), provided the window can be specified as an integrable function of log-frequency.

We performed a numerical analysis of the ``center of mass'' of the smoothed spectra produced by each method given the magnitude response of an analog band-pass filter (which is symmetric on a log-frequency scale) as the raw spectrum.
This analysis revealed that only the first method is unable to preserve the log-symmetry of the raw spectrum, as it shifts the center of mass upwards in frequency, resulting in a ``blue-shifted'' smoothed spectrum.
It is worth noting that this error is quite small for small smoothing bandwidths (e.g., the center of mass shifts by $< 1 \%$ of the center frequency for $1/3$-octave smoothing) and therefore may be tolerable in some applications.
We also smoothed a unit impulse with the first and third methods to explore how the maximum may shift after smoothing.
Results showed that only the first method shifts the maximum downwards in frequency, although we expect this phenomenon to be less significant for smaller smoothing bandwidths.

Regarding the computational cost of the proposed method, it is relevant to note that, for many smoothing windows, the definite integral of the window (see \eqnref{eq:A3_Smoothing_Weights:WeightsIntegral}) can be evaluated analytically, making the calculation of each weight sequence computationally inexpensive.
Furthermore, for all smoothing windows, the weighting function can be precomputed and stored in a matrix, recasting the smoothing operation as a matrix multiplication whose computational expense would be invariant with smoothing method, window, and bandwidth.

Future work should include an exploration of the equivalent time-domain implementation of the proposed method, wherein smoothing is expressed as multiplication of the input impulse response with frequency-dependent windows \citep{HatziantoniouMourjopoulos2000} and an investigation into the perceptual differences, if any, between the first and third methods for various smoothing bandwidths and in various applications (e.g., digital room correction, headphone equalization, etc.).